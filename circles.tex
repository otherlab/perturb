\documentclass[11pt]{article}
\usepackage{amsfonts,amssymb,amsthm,eucal,amsmath}
\usepackage{graphicx}
\usepackage[T1]{fontenc}
\usepackage{latexsym,url}
\usepackage{array}
\usepackage{subfig}
\usepackage{comment}
\usepackage{color}
\usepackage{hyperref}

\newcommand{\myspace}{\vspace{.1in}\noindent}
\newcommand{\mymyspace}{\vspace{.1in}}
\usepackage[inner=30mm, outer=30mm, textheight=225mm]{geometry}

\newtheorem{theorem}{Theorem}[section]
\newtheorem{prop}[theorem]{Proposition}
\newtheorem{corollary}[theorem]{Corollary}
\newtheorem{defn}[theorem]{Definition}
\newtheorem{notn}[theorem]{Notation}
\newtheorem{cond}[theorem]{Condition}
\newtheorem{ex}[theorem]{Example}
\newtheorem{rmk}[theorem]{Remark}
\newcommand{\co}{\negthinspace :}
\newcommand{\N}{\mathbb{N}}
\newcommand{\Z}{\mathbb{Z}}
\newcommand{\R}{\mathbb{R}}
\newcommand{\C}{\mathbb{C}}
\newcommand{\CP}{\mathbb{CP}}
\newcommand{\PSL}{\mathrm{PSL}_2(\mathbb{C})}
\newcommand{\area}{\operatorname{area}}
\newcommand{\rand}{\operatorname{rand}}
\newcommand{\diag}{\operatorname{diag}}
\newcommand{\nt}{\negthinspace}
\newcommand{\halpha}{\hat{\alpha}}
\newcommand{\hbeta}{\hat{\beta}}
\newcommand{\TODO}{{\color{red} TODO}}

\title{Circular arc predicates}
\author{Geoffrey Irving\thanks{Email: irving@naml.us, forrest@otherlab.com, Otherlab, San Francisco, CA, United States}
\and Forrest Green$^*$}
\date{Version 1, \today}

\begin{document}
\maketitle

\section{Circular arc predicates}

Here are most of the predicates required for circular arc Booleans in the plane.  $S_i = (c_i,r_i)$ be the $i$th
circle with center $c_i$ and radius $r_i$, and $c_{ij} = c_j - c_i$.
Note that all inequalities are strict; we do not need to worry about equalities, since by perturbation every variable is
algebraically independent of every other.  Our predicates are as follows:

\subsection{Do two circles intersect?}

Circles $S_0$ and $S_1$ intersect iff
$$ | r_1 - r_0 | < |c_{01}| < r_1 + r_0. $$
Since all quantities are positive, we can square to get
\begin{align*}
(r_1 - r_0)^2 < c_{01}^2 < (r_1 + r_0)^2
\end{align*}
where both inequalities are degree two.

\subsection{One intersection of two circles}

Given circles $S_0, S_1$, define the intersection point $x_{01}$ as the intersection
of $S_0,S_1$ to the left of line $c_0c_1$.  Parameterize the intersection by
\begin{align*}
x_{01} &= c_0 + \alpha c_{01} + \beta c_{01}^\perp.
\end{align*}
We have
\begin{align*}
(x_{01} - c_i)^2 &= r_i^2 \\
x_{01}^2 - 2x_{01} \cdot c_i + c_i^2 &= r_i^2.
\end{align*}
Subtracting the two circle equations gives
\begin{align*}
-2x_{01} \cdot c_{01} + c_1^2 - c_0^2 &= r_1^2 - r_0^2 \\
-2c_0 \cdot c_{01} -2\alpha c_{01}^2 + (c_0 + c_1) \cdot c_{01} &= r_1^2 - r_0^2 \\
(1-2\alpha) c_{01}^2 &= r_1^2 - r_0^2 \\
1 - 2 \alpha &= \frac{r_1^2 - r_0^2}{c_{01}^2} \\
\halpha = 2 c_{01}^2 \alpha &= c_{01}^2 + r_0^2 - r_1^2
\end{align*}
Substituting into $S_0$'s equation gives
\begin{align*}
(x_{01} - c_0)^2 &= r_0^2 \\
\left(\alpha c_{01} + \beta c_{01}^\perp \right)^2 &= r_0^2 \\
\alpha^2 c_{01}^2 + \beta^2 c_{01}^2 &= r_0^2 \\
\beta^2 &= \frac{r_0^2}{c_{01}^2} - \alpha^2 \\
\hbeta^2 = \left(2 c_{01}^2 \beta\right)^2 &= 4 r_0^2 c_{01}^2 - \halpha^2.
\end{align*}
To summarize, the intersection between circles $S_0$ and $S_1$ is described by
\begin{align*}
x_{01} &= c_0 + \alpha c_{01} + \beta c_{01}^\perp \\
\halpha = 2 \alpha c_{01}^2 &= c_{01}^2 - r_1^2 + r_0^2 \\
\hbeta^2 = (2c_{01}^2 \beta)^2 &= 4 r_0^2 c_{01}^2 - \halpha^2
\end{align*}
where $^\perp$ rotates left by $90^\circ$ and we choose the positive or negative square root for $\beta$ depending on which intersection is desired.

\subsection{Is one circle intersection above another?}

Given three circles $S_0,S_1,S_2$, is $x_{01}$ below $x_{02}$?  This predicate has the form
\begin{align*}
c_{0y} + \alpha_{01} c_{01y} + \beta_{01} c_{01x} &< c_{0y} + \alpha_{02} c_{02y} + \beta_{02} c_{02x} \\
0 &< \alpha_{02} c_{02y} - \alpha_{01} c_{01y} - \beta_{01} c_{01x} + \beta_{02} c_{02x} \\
0 &< \halpha_{02} c_{02y} c_{01}^2 - \halpha_{01} c_{01y} c_{02}^2 - \hbeta_{01} c_{01x} c_{02}^2+ \hbeta_{02} c_{02x} c_{01}^2 \\
0 &< A + B_1 \sqrt{C_1} + B_2 \sqrt{C_2}
\end{align*}
where $A,B_1,B_2,C_1,C_2$ are polynomials and $C_1, C_2 > 0$ since the two intersections are assumed to exist.  To reduce this equality to
purely polynomial equalities, we first compute the signs of $A, B_1, B_2$.  If these all match, we are done.  Otherwise we move the square root
terms that differ from $A$ in sign to the RHS and square.  Assuming $A > 0$, this gives either
\begin{align}
A + B_1 \sqrt{C_1} &> -B_2 \sqrt{C_2} \nonumber \\
A^2 + B_1^2 C_1 + 2 A B_1 \sqrt{C_1} &> B_2^2 C_2 \nonumber \\
A^2 + B_1^2 C_1 - B_2^2 C_2 &> -2 A B_1 \sqrt{C_1} \label{one-b}
\end{align}
or
\begin{align}
A &> -B_1 \sqrt{C_1} - B_2 \sqrt{C_2} \nonumber \\
A^2 &> B_1^2 C_1 + B_2^2 C_2 + 2 B_1 B_2 \sqrt{C_1 C_2} \nonumber \\
A^2 - B_1^2 C_1 - B_2^2 C_2 &> 2 B_1 B_2 \sqrt{C_1 C_2} \label{two-b}
\end{align}
The signs of the RHS's of (\ref{one-b}) and (\ref{two-b}) are known.  The polynomial LHS's are degree 10, but factor as
\begin{align*}
\begin{array}{@{}r} A^2 + B_1^2 C_1 - B_2^2 C_2 \phantom{\bigg(} \hspace{-.3em}= \\ \phantom{\bigg(} \end{array}&%
\begin{array}{@{}r@{}l@{}} c_{02}^2 \bigg(& c_{01}^2 \left(\halpha_{02} \left(\halpha_{02} c_{01}^2 - 2 \halpha_{01} c_{01y} c_{02y}\right)
  + 4 r_0^2 (c_{01x}^2 c_{02y}^2 - c_{01y}^2 c_{02x}^2)\right) \\
  &- \halpha_{01}^2 \left(c_{01x}^2 - c_{01y}^2\right) c_{02}^2 \bigg) \end{array} \\
\begin{array}{@{}r} A^2 - B_1^2 C_1 - B_2^2 C_2 \phantom{\bigg(} \hspace{-.3em}= \\ \phantom{\bigg(} \end{array}&%
\begin{array}{@{}r@{}l@{}} c_{01}^2 c_{02}^2 \bigg(& c_{02}^2 \halpha_{01}^2 + c_{01}^2 \halpha_{02}^2 - 2 c_{01y} c_{02y} \halpha_{01} \halpha_{02} \\
  &- 4 r_0^2 (c_{01y}^2 c_{02x}^2 + c_{01x}^2 c_{02y}^2 + 2c_{01x}^2 c_{02x}^2) \bigg) \end{array}
\end{align*}
and therefore reduce to degree 8 and 6, respectively.  If the LHS and RHS of (\ref{one-b}) or (\ref{two-b}) have the same sign, we square once
more to eliminate the final square root.  Assuming positive LHS, squaring (\ref{one-b}) gives
\begin{align*}
(A^2 + B_1^2 C_1 - B_2^2 C_2)^2 &> 4 A^2 B_1^2 C_1 \\
A^4 - 2A^2 B_1^2 C_1 + B_1^4 C_1^2 - 2 A^2 B_2^2 C_2 - 2 B_1^2 B_2^2 C_1 C_2 + B_2^4 C_2^2 &> 0 \\
E &> 0 
\end{align*}
and squaring (\ref{two-b}) gives
\begin{align*}
(A^2 - B_1^2 C_1 - B_2^2 C_2)^2 &> 4 B_1^2 B_2^2 C_1 C_2 \\
A^4 - 2A^2 B_1^2 C_1 + B_1^4 C_1^2 - 2 A^2 B_2^2 C_2 - 2 B_1^2 B_2^2 C_1 C_2 + B_2^4 C_2^2 &> 0 \\
E &> 0.
\end{align*}
That is, the two inequalities square into the same degree 20 polynomial $E$, which factors into degree $\le 6$ terms as
\begin{align*}
E &= c_{01}^4 c_{02}^4 E_+ E_- \\
E_\pm &= c_{02}^2 \halpha_{01}^2 + c_{01}^2 \halpha_{02}^2 - 2 \halpha_{01} \halpha_{02} (c_{01y} c_{02y} \pm c_{01x} c_{02x}) - 4 r_0^2 (c_{01x} c_{02y} \mp c_{01y} c_{02x})^2
\end{align*}

\subsection{Is the angle at an intersection counterclockwise?}

Let $t_0, t_1$ be the counterclockwise tangents at intersection $x_{01}$.  The following are equivalent:
\begin{align*}
t_0 \times t_1 &> 0 \\
(x_{01} - c_0)^\perp \times (x_{01} - c_1)^\perp &> 0 \\
(x_{01} - c_0) \times (x_{01} - c_1) &> 0 \\
(x_{01} - c_0) \times (c_0 - c_1) &> 0 \\
(\alpha c_{01} + \beta c_{01}^\perp) \times c_{01} &< 0 \\
\beta c_{01}^2 &> 0 \\
\hbeta &> 0
\end{align*}

Thus the angle is $> 180^\circ$ for the intersection to the left of segment $(c_0,c_1)$, $< 180^\circ$ for the intersection to the right.

\subsection{Is a circle intersection below a horizontal line?}

Consider two circles $S_0, S_1$ together with a horizontal line at coordinate $y$.  $x_{01}$ is below $y$ iff
\begin{align*}
y &> c_0^y + \alpha c_{01}^y + \beta c_{01}^x \\
2 (y-c_0^y) c_{01}^2 - \halpha c_{01}^y - \hbeta c_{01}^x &> 0
\end{align*}

\subsection{Does one circle-horizontal intersection occur to the left of another?}

Given a circle $S_0$ intersecting a horizontal line at coordinate $y$, the $x$ coordinate of the intersection satisfies
\begin{align*}
(x-c_0^x)^2 + (y-c_0^y)^2 &= r_0^2 \\
x &= c_0^x + \sqrt{r_0^2 - (y-c_0^y)^2}
\end{align*}
where the square root sign is $-$ for left and $+$ for right.

Consider two circles $S_0, S_1$ intersecting a horizontal line at coordinate $y$.  The first intersection is to the left of the second if
\begin{align*}
c_0^x + \sqrt{r_0^2 - (y-c_0^y)^2} &< c_1^x + \sqrt{r_1^2 - (y-c_1^y)^2} \\
0 &< c_{01}^x + \sqrt{r_1^2 - (y-c_1^y)^2} - \sqrt{r_0^2 - (y-c_0^y)^2}
\end{align*}

\section{Precision vs. flatness}

Unfortunately, implicit arcs using integer centers and radii are unable to represent straight lines exactly, raising an issue of precision.
For concreteness, let's assume an accuracy goal of 1 micron ($10^{-6}$ m) for a bounding box size of 1 meter.  This is a relative accuracy
requirement of $\epsilon = 10^{-6}$.  If we approximate a straight segment of length $l$ with a finite radius $r$, the maximum deviation is
\begin{align*}
\Delta &= r - \sqrt{r^2 - \left(\frac{l}{2}\right)^2} \\
       &= r - r \sqrt{1 - \frac{l^2}{4 r^2}} \\
       &\approx r - r \left(1 - \frac{l^2}{8 r^2} \right) \\
       &= \frac{l^2}{8 r}
\end{align*}
Requiring $\Delta < \epsilon l$, we have
\begin{align*}
\epsilon l &> \frac{l^2}{8 r} \\
\frac{r}{l} &> \frac{1}{8 \epsilon} \approx 10^{-5}
\end{align*}
This macroresolution requirement multiplies with the microresolution requirement, so if single segments are allowed to stretch all the way
across the domain, we'd require a total relative accuracy of $10^{-11}$,  Since $10^{-9}$ is right at the limit of what a single precision
int provides, this is impractical without switching to 64 bit.  $10^{-8}$ is probably the best we can do without running into integer overflow,
and this requires raising the integer limit to $2^{26}$ or so.  What can we get out of $10^{-8}$?  We can easily save one order of magnitude by
requiring no segment to extend more than $1/10$th of the bounding box, and then two orders of magnitude if the bounding box was assumed to be
only $10$ cm rather than 1 m.

It seems like 32 bits aren't working out, so we'll solve the problem by jumping to 64 bits generally.

\end{document}
