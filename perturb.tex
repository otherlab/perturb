\documentclass[11pt]{article}
\usepackage{amsfonts,amssymb,amsthm,eucal,amsmath}
\usepackage{graphicx}
\usepackage[T1]{fontenc}
\usepackage{latexsym,url}
\usepackage{array}
\usepackage{subfig}
\usepackage{comment}
\usepackage{color}

\newcommand{\myspace}{\vspace{.1in}\noindent}
\newcommand{\mymyspace}{\vspace{.1in}}
\usepackage[inner=30mm, outer=30mm, textheight=225mm]{geometry}

\newtheorem{theorem}{Theorem}[section]
\newtheorem{prop}[theorem]{Proposition}
\newtheorem{corollary}[theorem]{Corollary}
\newtheorem{defn}[theorem]{Definition}
\newtheorem{notn}[theorem]{Notation}
\newtheorem{cond}[theorem]{Condition}
\newtheorem{ex}[theorem]{Example}
\newtheorem{rmk}[theorem]{Remark}
\newcommand{\co}{\negthinspace :}
\newcommand{\N}{\mathbb{N}}
\newcommand{\Z}{\mathbb{Z}}
\newcommand{\R}{\mathbb{R}}
\newcommand{\C}{\mathbb{C}}
\newcommand{\CP}{\mathbb{CP}}
\newcommand{\PSL}{\mathrm{PSL}_2(\mathbb{C})}
\newcommand{\area}{\operatorname{area}}
\newcommand{\diag}{\operatorname{diag}}
\newcommand{\nt}{\negthinspace}
\newcommand{\TODO}{{\color{red} TODO}}

\title{A practical deterministic pseudorandom symbolic perturbation scheme for arbitrary polynomial predicates}
\author{Geoffrey Irving\thanks{Email: irving@naml.us, Otherlab, San Francisco, CA, United States}
\and Forrest Green\thanks{\TODO}}
\date{Version 1, \today}

\begin{document}
\maketitle

\begin{abstract}
We present a symbolic perturbation scheme for arbitrary polynomial geometric predicates, which combines the benefits of
Seidel's simple randomized linear perturbation scheme with Yap's multiple infinitesimal scheme for general predicates.
Like Seidel's randomized scheme, our method accepts black box polynomial functions as input.
For nonmaliciously chosen predicates, the algorithm is as fast as Seidel's scheme, and in particular scales reasonably
with the degree of the polynomial even for fully degenerate input.  Like Yap's scheme, the computed sign is deterministic,
never requiring an algorithmic restart (assuming a high quality pseudorandom generator), and works for arbitrary
predicates with no knowledge of their structure.  We also apply our technique to exactly or nearly exactly rounded
constructions that work correctly for degenerate input, using iterated l'H\^opital's rule to compute the necessary
singular limits.
\end{abstract}

\section{Notes}

\TODO: Mention that this stuff could be integrated into a library like CORE to work with arbitrary algebraic expressions.
It would be worse from a black box perspective, but still useful.

\TODO: Talk about different alternatives for detecting perfectly degenerate predicates via polynomial identity testing.  This is a key debugging tool.  In general, one major advantage of symbolic perturbation is that code that works for random input is highly likely to work for arbitrary input, in a much stronger sense than is normally true.  This is somewhat related to polynomial identity testing, in fact.
In detail, an algorithm is usually correct if it is correct for all semi-algebraic configurations of a relatively small number of points.  These test sets have strictly positive measure, so suitable random input will likely find them.

\TODO: General question: what is the best way to extend symbolic perturbation to functions with sqrts while preserving the black box structure?  Automatic differentiation, probably.

\TODO: Look up snap rounding
\TODO: Add Yap and Seidel citations, plus original simulation of simplicity paper
@article{seidel1998nature,

\cite{wein2006circles}: Exact by keeping everything as rational ``one-root'' numbers ($a + b \sqrt{c}$).  No discussion of degeneracies or degree.

\cite{halperin2004controlled}: Modify original circles to avoid degeneracies.  \TODO: Look up original controlled perturbation paper.  This is 

\cite{edelsbrunner1990simulation}: Original simulation of simplicity paper, followed \cite{yap1990symbolic}, \cite{emiris1992efficient}, and \cite{emiris1995general}.

\cite{yap1990symbolic}: Mentions that randomized schemes are unsatisfactory, though for a silly reason that they destroy the theoretical complexity.  The real reason is that a sufficiently
large problem may experience $O(1)$ expected degeneracies under a random perturbation, and therefore require a very large number of restarts, or the additional complexity of partial restarts.
They also discuss nonstandard analysis, which is the language we'll use to describe everything.  After dismissing randomized methods, Yap also notes that ``probably'' only a small number of
terms need to be evaluated, which is somewhat odd.  In general, Yap's discussion of the complexity of polynomial evaluation is completely broken, since it uses the number of monomials,
and the number of monomials is in general exponential.

\cite{emiris1992efficient}: Mentions both randomized linear schemes and $x_{i,j} + \epsilon i^j$ schemes, possibly mod some prime to avoid blowup.  Proofs of validity are required for
each new predicate to be used.  At least without the mod, the perturbation is nonrandom and therefore might explore horrible worst case configurations (of convex hull, say), as with Yap's
algorithm.  In fact yes: \cite{burnikel1994degeneracy} say that Emiris and Canny's scheme does hit this kind of worst case.

IMPORTANT: We should extract ourselves from the debate over whether perturbation is good or bad relative to other schemes, and simply present our new method and its advantages.
NOTE: If we're concerned that degeneracy handling will be slow, we can simply apply a randomized finite perturbation to the input.

\cite{emiris1995general}: Seems to be discussing the same randomized linear scheme.  Seidel clearly didn't invent this.

\cite{burnikel1994degeneracy}: Argues that perturbation is slower and more complicated to implement than simply handling degeneracies directly.  Interesting case: they note that \cite{emiris1992efficient}
can cause degenerate algorithms to hit worst case behavior of output dependent algorithms like convex hull.  Randomized schemes avoid this problem.


\cite{oruc2000explicit}, \cite{olver2006multivariate}, \cite{neidinger2009multivariable}: The full explicit factorization was cobbled together from these three papers.

\cite{halperin1998perturbation}, \cite{halperin2004controlled}: They add small perturbations to the input to guarantee no degeneracies, similar to what Martin does.
This is easier in their case because they have roughly constant radius spheres, and they have a nice classification of all the problems that can arise.  It does
require careful analysis of the predicates involved, and what kind of degeneracies can occur.  The second paper does a bunch of detailed floating point analysis.
It's possible these schemes are superior, but they require per-algorithm analysis.

\cite{devillers2012qualitative}: Instead of using algebraic techniques, use specially tailored geometric perturbations based on the type of predicates used.  This appears to
be significantly better in practice when it works, but requires per-predicate analysis.


\cite{devillers2000algebraic}: Degree 12 predicates for circular arc arrangements, no discussion of perturbation.

\section{Polynomial interpolation}

\TODO: Probably leave this out, and just list references.

The Vandermonde matrix required for polynomial interpolation admits a wonderful factorization, shown here for the degree 4 case:

\begin{align*}
\left(\begin{matrix}1 & x_{0} & x_{0}^{2} & x_{0}^{3}\\1 & x_{1} & x_{1}^{2} & x_{1}^{3}\\1 & x_{2} & x_{2}^{2} & x_{2}^{3}\\1 & x_{3} & x_{3}^{2} & x_{3}^{3}\end{matrix}\right)
=& \left(\begin{matrix}1&0&0&0\\\frac{1}{x_{0}-x_{1}}&\frac{1}{x_{1}-x_{0}}&0&0\\0&\frac{1}{x_{1}-x_{2}}&\frac{1}{x_{2}-x_{1}}&0\\0&0&\frac{1}{x_{2}-x_{3}}&\frac{1}{x_{3}-x_{2}}\end{matrix}\right)^{-1} \\
 & \left(\begin{matrix}1&0&0&0\\0&1&0&0\\0&\frac{1}{x_{0}-x_{2}}&\frac{1}{x_{2}-x_{0}}&0\\0&0&\frac{1}{x_{1}-x_{3}}&\frac{1}{x_{3}-x_{1}}\end{matrix}\right)^{-1} \\
 & \left(\begin{matrix}1&0&0&0\\0&1&0&0\\0&0&1&0\\0&0&\frac{1}{x_{0}-x_{3}}&\frac{1}{x_{3}-x_{0}}\end{matrix}\right)^{-1} \\
 & \left(\begin{matrix}1&x_{0}&0&0\\0&1&x_{1}&0\\0&0&1&x_{2}\\0&0&0&1\end{matrix}\right) \\
 & \left(\begin{matrix}1&0&0&0\\0&1&x_{0}&0\\0&0&1&x_{1}\\0&0&0&1\end{matrix}\right) \\
 & \left(\begin{matrix}1&0&0&0\\0&1&0&0\\0&0&1&x_{0}\\0&0&0&1\end{matrix}\right) 
\end{align*}

\section{Circular arc predicates}

\subsection{Do two circles intersect?}

Let's work out all the predicates we need for circular arc booleans in the plane.  Let $S_i = (c_i,r_i)$ be the $i$th
circle with center $c_i$ and radius $r_i$.  Circles $S_0$ and $S_1$ intersect iff
$$ | r_1 - r_0 | < |\Delta c| < r_1 + r_0. $$
Since all quantities are positive, we can square to get
\begin{align} \label{two-circles}
(r_1 - r_0)^2 < \Delta c^2 < (r_1 + r_0)^2
\end{align}
where both inequalities are degree two polynomial predicates.

\subsection{One intersection of two circles}

Given circles $S_0, S_1$, define the intersection point $x_{01}$ as the intersection
of $S_0,S_1$ to the left of line $c_0c_1$.  We have
\begin{align*}
x_{01} &= (1-\alpha) c_0 + \alpha c_1 + \beta \Delta c^\perp \\
(x_{01} - c_i)^2 &= r_i^2 \\
x_{01}^2 - 2x_{01} \cdot c_i + c_i^2 = r_i^2.
\end{align*}
Subtracting the two circle equations gives
\begin{align*}
-2x_{01} \cdot \Delta c + c_1^2 - c_0^2 = r_1^2 - r_0^2 \\
-2c_0 \cdot \Delta c -2\alpha {\Delta c}^2 + (c_0 + c_1) \cdot \Delta c &= r_1^2 - r_0^2 \\
(1-2\alpha) {\Delta c}^2 &= r_1^2 - r_0^2 \\
1 - 2 \alpha &= \frac{r_1^2 - r_0^2}{\Delta c^2}
\end{align*}
Let
$$\gamma = \frac{1 - 2\alpha}{2} = \frac{r_1^2 - r_0^2}{2\Delta c^2}$$
so that $\alpha = 1/2 - \gamma$ and
\begin{align*}
x_{01} &= (1-1/2 + \gamma) c_0 + (1/2 - \gamma) c_1 + \beta \Delta c^\perp \\
       &= \bar{c} - \gamma \Delta c + \beta \Delta c^\perp
\end{align*}
Substituting into $S_0$'s equation gives
\begin{align*}
(x_{01} - c_0)^2 &= r_0^2 \\
\left((1/2 - \gamma) \Delta c + \beta \Delta c^\perp \right)^2 &= r_0^2 \\
(1/2 - \gamma)^2 \Delta c^2 + \beta^2 \Delta c^2 &= r_0^2 \\
\beta^2 &= \frac{r_0^2}{\Delta c^2} - (1/2 - \gamma)^2 \\
\beta^2 &= \frac{r_0^2}{\Delta c^2} - \left(\frac{\Delta c^2 - r_1^2 + r_0^2}{2 \Delta c^2}\right)^2 \\
4 \beta^2 \Delta c^4 &= 4 r_0^2 \Delta c^2 - \left( \Delta c^2 - r_1^2 + r_0^2 \right)^2
\end{align*}
To recap, the intersection between circles $S_0$ and $S_1$ is described by
\begin{align*}
x_{01} &= c_0 + \alpha \Delta c + \beta \Delta c^\perp \\
       &= \bar{c} - \gamma \Delta c + \beta \Delta c^\perp \\
\hat{\alpha} = 2 \alpha \Delta c^2 &= \Delta c^2 - r_1^2 + r_0^2 \\
\hat{\gamma} = 2 \gamma \Delta c^2 &= r_1^2 - r_0^2 \\
\hat{\beta}^2 = 4 \beta^2 \Delta c^4 &= 4 r_0^2 \Delta c^2 - \hat{\alpha}^2
\end{align*}
where $^\perp$ rotates left by $90^\circ$ and we choose the positive square root for $\beta$.

\subsection{Is the intersection of two circles inside a third?}

Substituting $x_{01}$ into the equation for circle $S_2$ gives
\begin{align*}
(x_{01} - c_2)^2 &< r_2^2 \\
(c_0 + \alpha c_{01} + \beta c_{01}^\perp - c_2)^2 &< r_2^2 \\
(\alpha c_{01} + \beta c_{01}^\perp - c_{02})^2 &< r_2^2 \\
\alpha^2 c_{01}^2 + \beta^2 c_{01}^2 + c_{02}^2 - 2 \alpha c_{01}\cdot c_{02} - 2 \beta c_{01}\times c_{02} &< r_2^2 \\
4\alpha^2 c_{01}^4 + 4\beta^2 c_{01}^4 + 4c_{01}^2 c_{02}^2 - 8 \alpha c_{01}^2 c_{01}\cdot c_{02} - 8 \beta c_{01}^2 c_{01}\times c_{02} &< 4r_2^2 c_{01}^2 \\
\hat{\alpha}^2 + \hat{\beta}^2 + 4c_{01}^2 c_{02}^2 - 4 \hat{\alpha} c_{01}\cdot c_{02} - 4 \hat{\beta} c_{01}\times c_{02} &< 4r_2^2 c_{01}^2 \\
4 r_0^2 c_{01}^2 + 4 c_{01}^2 c_{02}^2 - 4 \hat{\alpha} c_{01}\cdot c_{02} - 4 \hat{\beta} c_{01}\times c_{02} &< 4r_2^2 c_{01}^2 \\
r_0^2 c_{01}^2 + c_{01}^2 c_{02}^2 - \hat{\alpha} c_{01}\cdot c_{02} - \hat{\beta} c_{01}\times c_{02} &< r_2^2 c_{01}^2 \\
c_{01}^2 (c_{02}^2 + r_0^2 - r_2^2) - \hat{\alpha} c_{01}\cdot c_{02} - \hat{\beta} c_{01}\times c_{02} &< 0 \\
\hat{\alpha}_{02} c_{01}^2 - \hat{\alpha}_{01} c_{01}\cdot c_{02} - \hat{\beta}_{01} c_{01}\times c_{02} &< 0 \\
\hat{\alpha}_{01} c_{01}\cdot c_{02} - \hat{\alpha}_{02} c_{01}^2 + \hat{\beta}_{01} c_{01}\times c_{02} &> 0 \\
\end{align*}

\subsection{Are the intersections of two circles with a third counterclockwise?}

Alternatively, given three circles $S_0, S_1, S_2$, we ask whether $x_{01}$ occurs to the right of $x_{02}$.  Equivalently, do we have
\begin{align*}
(x_{01} - c_0) \times (x_{02} - c_0) &> 0 \\
(\alpha_1 \Delta c_1 + \beta_1 \Delta c_1^\perp) \times (\alpha_2 \Delta c_2 + \beta_2 \Delta c_2^\perp) &> 0 \\
(\alpha_1 \alpha_2 + \beta_1 \beta_2) \Delta c_1 \times \Delta c_2 + (\alpha_1 \beta_2 - \beta_1 \alpha_2) \Delta c_1 \cdot \Delta c_2 &> 0
\end{align*}
Multiply through by $4 \Delta c_1^2 \Delta c_2^2$ to clear nonconstant fractions to get
\begin{align} \label{circle-order}
(\alpha'_1 \alpha'_2 + \beta'_1 \beta'_2) \Delta c_1 \times \Delta c_2 + (\alpha'_1 \beta'_2 - \beta'_1 \alpha'_2) \Delta c_1 \cdot \Delta c_2 &> 0
\end{align}
This has the form
$$a_6 + a_4 \sqrt{b_4} + a_4 \sqrt{b_4} + a_2 \sqrt{b_4} \sqrt{b_4}$$
where subscript $k$ denotes degree $k$.  This matches Lemma (5) from "On the Degree of Standard Geometric Predicates for Line Transversals in 3D", so the final predicate degree is 24.

Finally, we need a much simpler version of (\ref{circle-order}) where the second direction is one of the coordinate axes, call it $e$:
\begin{align*}
e \times (x_{01} - c_0) &> 0 \\
e \times (\alpha \Delta c + \beta \Delta c^\perp) &> 0 \\
e \times (\hat{\alpha} \Delta c + \hat{\beta} \Delta c^\perp) &> 0 \\
\hat{\alpha} e \times \Delta c &> -\hat{\beta} e \cdot \Delta c \\
\hat{\alpha}^2 (e \times \Delta c)^2 &> \hat{\beta}^2 (e \cdot \Delta c)
\end{align*}
where the last equation is correct only for appropriate sign configurations.

\subsection{Is one circle intersection above another?}

Given four circles $S_0$ through $S_3$, is $x_{01}$ below $x_{23}$?  This predicate has the form
\begin{align*}
c_0^y + \alpha_{01} c_{01}^y + \beta_{01} c_{01}^x &< c_2^y + \alpha_{23} c_{23}^y + \beta_{23} c_{23}^x \\
0 &< c_{02}^y + \alpha_{23} c_{23}^y - \alpha_{01} c_{01}^y + \beta_{23} c_{23}^x - \beta_{01} c_{01}^x \\
0 &< 2 c_{02}^y c_{01}^2 c_{23}^2 + \hat{\alpha}_{23} c_{23}^y c_{01}^2 - \hat{\alpha}_{01} c_{01}^y c_{23}^2 + \hat{\beta}_{23} c_{23}^x c_{01}^2 - \hat{\beta}_{01} c_{01}^x c_{23}^2
\end{align*}

\subsection{Does the horizontal line through one circle intersection intersect another circle?}

In other words, given circles $S_0, S_1, S_2$, do we have
\begin{align*}
c_2^y - r_2 &< c_0^y + \alpha c_{01}^y + \beta c_{01}^x < c_2^y + r_2 \\
2 (c_{02}^y - r_2) c_{01}^2 &< \hat{\alpha} c_{01}^y + \hat{\beta} c_{01}^x < 2 (c_{02}^y + r_2) c_{01}^2
\end{align*}

\subsection{Is the intersection of a horizontal line through one circle intersection with another circle to the right of the first intersection?}

Given three circles $S_0, S_1, S_2$, where the horizontal line through $x_{01}$ intersects $S_2$ at $p$, we ask whether $x_{01}^x < p^x$.  We have
\begin{align*}
p^y &= x_{01}^x \\
(p - c_2)^2 &= r_2^2 \\
(p^x - c_2^x)^2 + (x_{01}^y - c_2^y)^2 &= r_2^2 \\
p^x - c_2^x &= s \sqrt{r_2^2 - (x_{01}^y - c_2^y)^2} \\
p^x &= c_2^x + s \sqrt{r_2^2 - (x_{01}^y - c_2^y)^2}
\end{align*}
where $s = \pm 1$ depending on which intersection we're considering.  Our predicate is
\begin{align*}
x_{01}^x < c_2^x + s \sqrt{r_2^2 - (x_{01}^y - c_2^y)^2} \\
x_{01}^x - c_2^x < s \sqrt{r_2^2 - (x_{01}^y - c_2^y)^2}
\end{align*}
Assuming both sides are positive, squaring gives
\begin{align*}
(x_{01}^x - c_2^x)^2 &< r_2^2 - (x_{01}^y - c_2^y)^2 \\
(x_{01}^x - c_2^x)^2 + (x_{01}^y - c_2^y)^2 &< r_2^2 \\
(x_{01} - c_2)^2 &< r_2^2
\end{align*}
which is exactly the question of whether $x_{01}$ is inside $S_2$ as answered above.

Before squaring, the sign of the LHS is given by
\begin{align*}
x_{01}^x - c_2^x &> 0 \\
\alpha c_{01}^x + \beta c_{01}^y - c_2^x &> 0 \\
\hat{\alpha} c_{01}^x + \hat{\beta} c_{01}^y - 2 c_2^x c_{01}^2 &> 0
\end{align*}

\subsection{Is the angle at an intersection counterclockwise?}

Let $t_0, t_1$ be the counterclockwise tangents at intersection $x_{01}$.  The following are equivalent:
\begin{align*}
t_0 \times t_1 &> 0 \\
(x_{01} - c_0)^\perp \times (x_{01} - c_1)^\perp &> 0 \\
(x_{01} - c_0) \times (x_{01} - c_1) &> 0 \\
(x_{01} - c_0) \times (c_0 - c_1) &> 0 \\
(\alpha \Delta c + \beta \Delta c^\perp) \times \Delta c &< 0 \\
\beta \Delta c^2 &> 0 \\
\hat{\beta} &> 0
\end{align*}

Thus the angle is $> 180^\circ$ for the intersection to the left of segment $(c_0,c_1)$, $< 180^\circ$ for the intersection to the right.

\section{Circular arc polygons}

Well, those are all the predicates we need for exact circular arc constructive solid geometry.  \TODO: Are those really all we needed?

\section{Precision vs. flatness}

Unfortunately, implicit arcs using integer centers and radii are unable to represent straight lines exactly, raising an issue of precision.
For concreteness, let's assume an accuracy goal of 1 micron ($10^{-6}$ m) for a bounding box size of 1 meter.  This is a relative accuracy
requirement of $\epsilon = 10^{-6}$.  If we approximate a straight segment of length $l$ with a finite radius $r$, the maximum deviation is
\begin{align*}
\Delta &= r - \sqrt{r^2 - \left(\frac{l}{2}\right)^2} \\
       &= r - r \sqrt{1 - \frac{l^2}{4 r^2}} \\
       &\approx r - r \left(1 - \frac{l^2}{8 r^2} \right) \\
       &= \frac{l^2}{8 r}
\end{align*}
Requiring $\Delta < \epsilon l$, we have
\begin{align*}
\epsilon l &> \frac{l^2}{8 r} \\
\frac{r}{l} &> \frac{1}{8 \epsilon} \approx 10^{-5}
\end{align*}
This macroresolution requirement multiplies with the microresolution requirement, so if single segments are allowed to stretch all the way
across the domain, we'd require a total relative accuracy of $10^{-11}$,  Since $10^{-9}$ is right at the limit of what a single precision
int provides, this is impractical without switching to 64 bit.  $10^{-8}$ is probably the best we can do without running into integer overflow,
and this requires raising the integer limit to $2^{26}$ or so.  What can we get out of $10^{-8}$?  We can easily save one order of magnitude by
requiring no segment to extend more than $1/10$th of the bounding box, and then two orders of magnitude if the bounding box was assumed to be
only $10$ cm rather than 1 m.

So that's what we'll do, at least for now:
\begin{itemize}
\item Ask for a relative accuracy of $10^-5$, corresponding to 1 micron in a 10 cm box.
\item Subdivide arcs down to 10\% of the bounding box size.
\item Raise the single precision integer limit from $2^{24}$ to $2^{26}$, abandoning the ``fits within a float'' invariant.
\end{itemize}
Additionally, our first implementation will ignore the subdivision step, and therefore achieve only $10^{-4.5}$ accuracy.  All of this stuff
does leave a bit of a bad taste in one's mouth, but I'll set that aside for now.

\section{Safe quantization}

Suppose we quantize a single circular arc polygon to integer centers $c_i$ and radii $r_i$.  The contour makes sense only if adjacent circles
actually intersect, so we require
\begin{align*}
|r_{i+1} - r_i| < |\Delta c_i| < r_i + r_{i+1}
\end{align*}
While it's possible to achieve this, it'd be hard to guarantee that the intersections don't move a lot.

\section{Endpoint-based arc representation}

Just in case, let's talk about endpoint based arc representation.  For stability, we'll arrange things so that arcs subtend under $180^\circ$.
Thus, arc $(x_0,x_1,x_2)$ moves from $x_0$ to $x_2$ around the circumcircle in whichever way is fastest.

\subsection{Implicitization}

Assuming general position, which we can of course always do, the implicit equation for circle $(x_0,x_1,x_2)$ is
\begin{align*}
\det \left(\begin{array}{ccc}
  x^2 & x & 1 \\
  x_0^2 & x_0 & 1 \\
  x_1^2 & x_1 & 1 \\
  x_2^2 & x_2 & 1
\end{array} \right ) &= 0 \\
a_2 x^2 + a_1 \cdot x + a_0 &= 0
\end{align*}
where $a_2, a_1, a_0$ have degrees $2, 3, 4$, respectively.  Note: $a_2$ is twice the area of triangle $(x_0,x_1,x_2)$.  Expanding the equation for
a circle, we have
\begin{align*}
(x - c)^2 - r^2 &= 0 \\
x^2 - 2c \cdot x + c^2 - r^2 &= 0 \\
a_2 x^2 - 2a_2 c \cdot x + a_2 (c^2 - r^2) &= 0 \\
\hat{c} = 2 a_2 c &= -a_1 \\
a_2 r^2 &= a_2 c^2 - a_0 \\
\hat{r}^2 = 4 a_2^2 r^2 &= 4 a_2^2 c^2 - 4 a_2 a_0 = a_1^2 - 4 a_2 a_0 \\
\end{align*}
A rough estimate is that our $24$ degree intersection ordering predicate would become $4(6+12) = 72$ degree.

\bibliography{references}
\bibliographystyle{acm}
%\bibliographystyle{model1-num-names}
\end{document}
