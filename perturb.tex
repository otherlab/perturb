\documentclass[11pt]{article}
\usepackage{amsfonts,amssymb,amsthm,eucal,amsmath}
\usepackage{graphicx}
\usepackage[T1]{fontenc}
\usepackage{latexsym,url}
\usepackage{array}
\usepackage{subfig}
\usepackage{comment}
\usepackage{color}

\newcommand{\myspace}{\vspace{.1in}\noindent}
\newcommand{\mymyspace}{\vspace{.1in}}
\usepackage[inner=30mm, outer=30mm, textheight=225mm]{geometry}

\newtheorem{theorem}{Theorem}[section]
\newtheorem{prop}[theorem]{Proposition}
\newtheorem{corollary}[theorem]{Corollary}
\newtheorem{defn}[theorem]{Definition}
\newtheorem{notn}[theorem]{Notation}
\newtheorem{cond}[theorem]{Condition}
\newtheorem{ex}[theorem]{Example}
\newtheorem{rmk}[theorem]{Remark}
\newcommand{\co}{\negthinspace :}
\newcommand{\N}{\mathbb{N}}
\newcommand{\Z}{\mathbb{Z}}
\newcommand{\R}{\mathbb{R}}
\newcommand{\C}{\mathbb{C}}
\newcommand{\CP}{\mathbb{CP}}
\newcommand{\PSL}{\mathrm{PSL}_2(\mathbb{C})}
\newcommand{\area}{\operatorname{area}}
\newcommand{\diag}{\operatorname{diag}}
\newcommand{\nt}{\negthinspace}
\newcommand{\TODO}{{\color{red} TODO}}

\title{A practical deterministic pseudorandom symbolic perturbation scheme for arbitrary polynomial predicates}
\author{Geoffrey Irving\thanks{Email: irving@naml.us, Otherlab, San Francisco, CA, United States}
\and Forrest Green\thanks{\TODO}}
\date{Version 1, \today}

\begin{document}
\maketitle

\begin{abstract}
We present a symbolic perturbation scheme for arbitrary polynomial geometric predicates, which combines the benefits of
Seidel's simple randomized linear perturbation scheme with Yap's multiple infinitesimal scheme for general predicates.
Like Seidel's randomized scheme, our method accepts black box polynomial functions as input.
For nonmaliciously chosen predicates, the algorithm is as fast as Seidel's scheme, and in particular scales linearly
with the degree of the polynomial even for fully degenerate input.  Like Yap's scheme, the computed sign is deterministic,
thus never requires an algorithmic restart (assuming a high quality pseudorandom generator), and works for arbitrary
predicates with no knowledge of their structure.  We also apply our technique to exactly or nearly exactly rounded
constructions that work correctly for degenerate input, using iterated l'H\^opital's rule to compute the necessary
singular limits.
\end{abstract}

\TODO: Mention that this stuff could be integrated into a library like CORE to work with arbitrary algebraic expressions.
It would be worse from a black box perspective, but still useful.

\TODO: Talk about different alternatives for detecting perfectly degenerate predicates via polynomial identity testing.  This is a key debugging tool.  In general, one major advantage of symbolic perturbation is that code that works for random input is highly likely to work for arbitrary input, in a much stronger sense than is normally true.  This is somewhat related to polynomial identity testing, in fact.
In detail, an algorithm is usually correct if it is correct for all semi-algebraic configurations of a relatively small number of points.  These test sets have strictly positive measure, so suitable random input will likely find them.

\TODO: General question: what is the best way to extend symbolic perturbation to functions with sqrts while preserving the black box structure?  Automatic differentiation, probably.

\section{Circular arc predicates}

\subsection{Do two circles intersect?}

Let's work out all the predicates we need for circular arc booleans in the plane.  Let $S_i = (c_i,r_i)$ be the $i$th
circle with center $c_i$ and radius $r_i$.  Circles $S_0$ and $S_1$ intersect iff
$$ | r_1 - r_0 | < |\Delta c| < r_1 + r_0. $$
Since all quantities are positive, we can square to get
\begin{align} \label{two-circles}
(r_1 - r_0)^2 < \Delta c^2 < (r_1 + r_0)^2
\end{align}
where both inequalities are degree two polynomial predicates.

\subsection{One intersection of two circles}

Given circles $S_0, S_1$, define the intersection point $x_{01}$ as the intersection
of $S_0,S_1$ to the left of line $c_0c_1$.  We have
\begin{align*}
x_{01} &= (1-\alpha) c_0 + \alpha c_1 + \beta \Delta c^\perp \\
(x_{01} - c_i)^2 &= r_i^2 \\
x_{01}^2 - 2x_{01} \cdot c_i + c_i^2 = r_i^2.
\end{align*}
Subtracting the two circle equations gives
\begin{align*}
-2x_{01} \cdot \Delta c + c_1^2 - c_0^2 = r_1^2 - r_0^2 \\
-2c_0 \cdot \Delta c -2\alpha {\Delta c}^2 + (c_0 + c_1) \cdot \Delta c &= r_1^2 - r_0^2 \\
(1-2\alpha) {\Delta c}^2 &= r_1^2 - r_0^2 \\
1 - 2 \alpha &= \frac{r_1^2 - r_0^2}{\Delta c^2}
\end{align*}
Let
$$\gamma = \frac{1 - 2\alpha}{2} = \frac{r_1^2 - r_0^2}{2\Delta c^2}$$
so that $\alpha = 1/2 - \gamma$ and
\begin{align*}
x_{01} &= (1-1/2 + \gamma) c_0 + (1/2 - \gamma) c_1 + \beta \Delta c^\perp \\
       &= \bar{c} - \gamma \Delta c + \beta \Delta c^\perp
\end{align*}
Substituting into $S_0$'s equation gives
\begin{align*}
(x_{01} - c_0)^2 &= r_0^2 \\
\left((1/2 - \gamma) \Delta c + \beta \Delta c^\perp \right)^2 &= r_0^2 \\
(1/2 - \gamma)^2 \Delta c^2 + \beta^2 \Delta c^2 &= r_0^2 \\
\beta^2 &= \frac{r_0^2}{\Delta c^2} - (1/2 - \gamma)^2 \\
\beta^2 &= \frac{r_0^2}{\Delta c^2} - \left(\frac{\Delta c^2 - r_1^2 + r_0^2}{2 \Delta c^2}\right)^2 \\
4 \beta^2 \Delta c^4 &= 4 r_0^2 \Delta c^2 - \left( \Delta c^2 - r_1^2 + r_0^2 \right)^2
\end{align*}
To recap, the intersection between circles $S_0$ and $S_1$ is described by
\begin{align*}
x_{01} &= c_0 + \alpha \Delta c + \beta \Delta c^\perp \\
       &= \bar{c} - \gamma \Delta c + \beta \Delta c^\perp \\
\hat{\alpha} = 2 \alpha \Delta c^2 &= \Delta c^2 - r_1^2 + r_0^2 \\
\hat{\gamma} = 2 \gamma \Delta c^2 &= r_1^2 - r_0^2 \\
\hat{\beta}^2 = 4 \beta^2 \Delta c^4 &= 4 r_0^2 \Delta c^2 - \hat{\alpha}^2
\end{align*}
where $^\perp$ rotates left by $90^\circ$ and we choose the positive square root for $\beta$.

\subsection{Is the intersection of two circles inside a third?}

Substituting $x_{01}$ into the equation for circle $S_2$ gives
\begin{align*}
(x_{01} - c_2)^2 &< r_2^2 \\
(\bar{c} - \gamma \Delta c + \beta \Delta c^\perp - c_2)^2 &< r_2^2 \\
u &= \bar{c} - c_2 \\
u^2 - \gamma^2 \Delta c^2 + \beta^2 \Delta c^2 - 2\gamma u \cdot \Delta c + 2\beta u \times \Delta c &< r_2^2 \\
u^2 - \gamma^2 \Delta c^2 + \beta^2 \Delta c^2 - 2\gamma u \cdot \Delta c - r_2^2 &< -2\beta u \times \Delta c
\end{align*}
The LHS involves a few fractions, which we clear by multiplying the entire equation by $4\Delta c^2$:
\begin{align*}
4u^2 \Delta c^2 - 4\gamma^2 \Delta c^4 + 4\beta^2 \Delta c^4 - 8\gamma \Delta c^2 (u \cdot \Delta c) - 4 r_2^2 \Delta c^2 &< -8\beta \Delta c^2 (u \times \Delta c) \\
v^2 \Delta c^2 - \hat{\gamma}^2 + \hat{\beta}^2 - 2\hat{\gamma} (v \cdot \Delta c) - 4 r_2^2 \Delta c^2 &< -4 \hat{\beta} (v \times \Delta c)
\end{align*}
where $v = 2u = c_0+c_1-2c_2$.
The LHS is now an integer polynomial, but the RHS involves a positive square root.  Thus we first check the signs of the LHS and the non-sqrt portion of the RHS; if they differ, we
are done.  If they are both positive, squaring preserves the $<$, and we have
\begin{align} 
\left( v^2 \Delta c^2 - \hat{\gamma}^2 + \hat{\beta}^2 - 2\hat{\gamma} (v \cdot \Delta c) - 4 r_2^2 \Delta c^2 \right)^2 &< 16 \hat{\beta}^2 (v \times \Delta c)^2
\label{three-circles}
\end{align}
which is an 8th degree integer polynomial in the centers and radii.  The both negative case is similar, but $<$ becomes $>$.

\subsection{Are the intersections of two circles with a third counterclockwise?}

Alternatively, given three circles $S_0, S_1, S_2$, we ask whether $x_{01}$ occurs to the right of $x_{02}$.  Equivalently, do we have
\begin{align*}
(x_{01} - c_0) \times (x_{02} - c_0) &> 0 \\
(\alpha_1 \Delta c_1 + \beta_1 \Delta c_1^\perp) \times (\alpha_2 \Delta c_2 + \beta_2 \Delta c_2^\perp) &> 0 \\
(\alpha_1 \alpha_2 + \beta_1 \beta_2) \Delta c_1 \times \Delta c_2 + (\alpha_1 \beta_2 - \beta_1 \alpha_2) \Delta c_1 \cdot \Delta c_2 &> 0
\end{align*}
Multiply through by $4 \Delta c_1^2 \Delta c_2^2$ to clear nonconstant fractions to get
\begin{align} \label{circle-order}
(\alpha'_1 \alpha'_2 + \beta'_1 \beta'_2) \Delta c_1 \times \Delta c_2 + (\alpha'_1 \beta'_2 - \beta'_1 \alpha'_2) \Delta c_1 \cdot \Delta c_2 &> 0
\end{align}
This has the form
$$a_6 + a_4 \sqrt{b_4} + a_4 \sqrt{b_4} + a_2 \sqrt{b_4} \sqrt{b_4}$$
where subscript $k$ denotes degree $k$.  This matches Lemma (5) from "On the Degree of Standard Geometric Predicates for Line Transversals in 3D", so the final predicate degree is 24.

Finally, we need a much simpler version of (\ref{circle-order}) where the second direction is one of the coordinate axes, call it $e$:
\begin{align*}
e \times (x_{01} - c_0) &> 0 \\
e \times (\alpha \Delta c + \beta \Delta c^\perp) &> 0 \\
e \times (\hat{\alpha} \Delta c + \hat{\beta} \Delta c^\perp) &> 0 \\
\hat{\alpha} e \times \Delta c &> -\hat{\beta} e \cdot \Delta c \\
\hat{\alpha}^2 (e \times \Delta c)^2 &> \hat{\beta}^2 (e \cdot \Delta c)
\end{align*}
where the last equation is correct only for appropriate sign configurations.

\subsection{Is one circle intersection above another?}

Given four circles $S_0$ through $S_3$, is $x_{01}$ below $x_{23}$?  This predicate has the form
\begin{align*}
c_0^y + \alpha_{01} c_{01}^y + \beta_{01} c_{01}^x &< c_2^y + \alpha_{23} c_{23}^y + \beta_{23} c_{23}^x \\
0 &< c_{02}^y + \alpha_{23} c_{23}^y - \alpha_{01} c_{01}^y + \beta_{23} c_{23}^x - \beta_{01} c_{01}^x \\
0 &< 2 c_{02}^y c_{01}^2 c_{23}^2 + \hat{\alpha}_{23} c_{23}^y c_{01}^2 - \hat{\alpha}_{01} c_{01}^y c_{23}^2 + \hat{\beta}_{23} c_{23}^x c_{01}^2 - \hat{\beta}_{01} c_{01}^x c_{23}^2
\end{align*}

\subsection{Does the horizontal line through one circle intersection intersect another circle?}

In other words, given circles $S_0, S_1, S_2$, do we have
\begin{align*}
c_2^y - r_2 &< c_0^y + \alpha c_{01}^y + \beta c_{01}^x < c_2^y + r_2 \\
2 (c_{02}^y - r_2) c_{01}^2 &< \hat{\alpha} c_{01}^y + \hat{\beta} c_{01}^x < 2 (c_{02}^y + r_2) c_{01}^2
\end{align*}

\subsection{Is the intersection of a horizontal line through one circle intersection with another circle to the right of the first intersection?}

Given three circles $S_0, S_1, S_2$, where the horizontal line through $x_{01}$ intersects $S_2$ at $p$, we ask whether $x_{01}^x < p^x$.  We have
\begin{align*}
p^y &= x_{01}^x \\
(p - c_2)^2 &= r_2^2 \\
(p^x - c_2^x)^2 + (x_{01}^y - c_2^y)^2 &= r_2^2 \\
p^x - c_2^x &= s \sqrt{r_2^2 - (x_{01}^y - c_2^y)^2} \\
p^x &= c_2^x + s \sqrt{r_2^2 - (x_{01}^y - c_2^y)^2}
\end{align*}
where $s = \pm 1$ depending on which intersection we're considering.  Our predicate is
\begin{align*}
x_{01}^x < c_2^x + s \sqrt{r_2^2 - (x_{01}^y - c_2^y)^2} \\
x_{01}^x - c_2^x < s \sqrt{r_2^2 - (x_{01}^y - c_2^y)^2}
\end{align*}
Assuming both sides are positive, squaring gives
\begin{align*}
(x_{01}^x - c_2^x)^2 &< r_2^2 - (x_{01}^y - c_2^y)^2 \\
(x_{01}^x - c_2^x)^2 + (x_{01}^y - c_2^y)^2 &< r_2^2 \\
(x_{01} - c_2)^2 &< r_2^2 \\
(\alpha c_{01} + \beta c_{01}^\perp - c_{02})^2 &< r_2^2 \\
\alpha^2 c_{01}^2 + \beta^2 c_{01}^2 + c_{02}^2 - 2 \alpha c_{01} \cdot c_{02} + 2 \beta c_{01} \times c_{02} &< r_2^2 \\
\hat{\alpha}^2 + \hat{\beta}^2 + 4 c_{02}^2 c_{01}^2 - 4\hat{\alpha} c_{01} \cdot c_{02} + 4 \hat{\beta} c_{01} \times c_{02} &< 4 r_2^2 c_{01}^2 \\
4 r_0^2 c_{01}^2 + 4 c_{02}^2 c_{01}^2 - 4\hat{\alpha} c_{01} \cdot c_{02} + 4 \hat{\beta} c_{01} \times c_{02} &< 4 r_2^2 c_{01}^2 \\
4 c_{01}^2 (r_0^2 + c_{02}^2 - r_2^2) - 4\hat{\alpha} c_{01} \cdot c_{02} + 4 \hat{\beta} c_{01} \times c_{02} &< 0 \\
c_{01}^2 (r_0^2 + c_{02}^2 - r_2^2) - \hat{\alpha} c_{01} \cdot c_{02} + \hat{\beta} c_{01} \times c_{02} &< 0 
\end{align*}

\TODO: This is exactly the question of whether $x_{01}$ is inside $S_2$, and should be merged into that section.

Before squaring, the sign of the LHS is given by
\begin{align*}
x_{01}^x - c_2^x &> 0 \\
\alpha c_{01}^x + \beta c_{01}^y - c_2^x &> 0 \\
\hat{\alpha} c_{01}^x + \hat{\beta} c_{01}^y - 2 c_2^x c_{01}^2 &> 0
\end{align*}

\subsection{Is the angle at an intersection counterclockwise?}

Let $t_0, t_1$ be the counterclockwise tangents at intersection $x_{01}$.  The following are equivalent:
\begin{align*}
t_0 \times t_1 &> 0 \\
(x_{01} - c_0)^\perp \times (x_{01} - c_1)^\perp &> 0 \\
(x_{01} - c_0) \times (x_{01} - c_1) &> 0 \\
(x_{01} - c_0) \times (c_0 - c_1) &> 0 \\
(\alpha \Delta c + \beta \Delta c^\perp) \times \Delta c &< 0 \\
\beta \Delta c^2 &> 0 \\
\hat{\beta} &> 0
\end{align*}

Thus the angle is $> 180^\circ$ for the intersection to the left of segment $(c_0,c_1)$, $< 180^\circ$ for the intersection to the right.

\section{Circular arc polygons}

Well, those are all the predicates we need for exact circular arc constructive solid geometry.  \TODO: Are those really all we needed?

\end{document}
